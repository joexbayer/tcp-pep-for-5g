\documentclass[a4paper,english, 11pt]{report}
\usepackage[utf8]{inputenc}
\usepackage[T1]{fontenc}
\usepackage{uiomasterfp}
\usepackage{url}
\author{Joe Bayer}
\title{TCP PEP}
\subtitle{Extension of a TCP Performance Enhancing Proxy to
Support Non-interactive Applications}

\begin{document}
\uiomasterfp[program={Informatics: Programming and System Architecture}, supervisors={Michael Welzl\and Kristjon Ciko}]

\tableofcontents

\chapter{Intro}

\chapter{Background}

\section{Future of wireless communication.}
\subsection{5G}
The future of wireless communication has seen a lot of improvements such as...
... highly increased bandwidth ... using Millimeter frequency bands ... but at the cost of Highly fluctuating bandwidth with wireless networks, especially with mmWave.\\

\subsection{mmWave}s
What is mmWave? Why? How?
A big problem with mmWave communication is Line of sight blocking (signal path blocking). This is were ...
Even the human body can create enough blockage to drastically reduce the bandwidth. 

[Figure from mmwave paper showing bandwith fluctuations.]

''To achieve high throughput as well as low latency, these
wireless networks will rely heavily on millimeter wave frequency bands (30-300 GHz), due to the large amounts of
spectrum available on those bands.'' (qoute mmwave paper)

''Applications that require extremely low latency are
expected to be a major driver of 5G and WLAN networks that
include millimeter wave (mmWave) links.'' (qoute mmwave paper)

''Verizon’s mmWave network deployed in
Minneapolis and Chicago reported a high handover frequency
due to frequent disruptions in mmWave connectivity''(cite A First Look at Commercial 5G Performance on Smartphones)

\subsection{Buffer bloat. (Buffering)}
Large buffers, create buffer bloat, which is detrimental for latency sensitive applications. Preferred to drop packets and keep buffers small to avoid buffering time sensitive packets such as SYN packets.

Most focus has been on (helping? Supporting?) latency sensitive applications like virtual reality or remote surgery to name a few. This thesis will explore non-interactive applications where latency is not that critical and more buffering is acceptable and most likely desirable.

\subsection{Non-Interactive Applications}
Non-Interactive applications such as Web traffic, File transfers and Videos? can benefit from larger buffering, especially with fluctuating bandwidths. Being able to have packets buffered for when the bandwidth is high will decrease delay times. (need citation or prove it myself?) 

\section{TCP/IP}

Interactive traffic uses TCP? {source}
End to end argument.
TCP handshake, reduce RTTs but using TCP Fast Open.
\\
End to End congestion controller not very suited for highly fluctuating bandwidth.(cite David Hayes?)

\subsection{TCP Fast Open}
Short flows terminating in a few round-trips. Meaning the "bottleneck" is the required initial TCP handshake.\\
TCP Fast Open allows data being exchanged during the handshake. 

\subsection{0 RTT}
Allow "syn fowarding" with TCP Fast Open creating a 0RTT increase when connecting through a proxy.
0RTT Transport Converter ~\cite{rfc8803}.

\subsection{Congestion control}
Congestion controller domains (different congestion controllers.)~\cite{rfc5783}.
Wireless versus Wired networks, different congestion controllers needs.
But domain split congestion control is not able to adapt fast enough to large changes in bandwidth. (cite David?)


\section{PEPs}
More logic inside the networks. Domain splitting and 0RTT. {cite Kristjon Ciko?}

\section{Kernel Modules}
LKM (Loadable Kernel Modules), "program" running inside the Linux kernel.
Userspace vs Kernel.
\subsection{System Calls}
Reduce over head from userspace -> kernel system calls.

\chapter{Implementation | Design}
\chapter{Evaluation}
\chapter{Conclusion}

\bibliography{myBib.bib}{}
\bibliographystyle{plain}
\end{document}s