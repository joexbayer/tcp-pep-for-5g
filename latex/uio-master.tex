\documentclass[a4paper,english, 11pt]{report}
\usepackage[utf8]{inputenc}
\usepackage[T1]{fontenc}
\usepackage{uiomasterfp}
\usepackage{url}
\author{Joe Bayer}
\title{TCP PEP}
\subtitle{Extension of a TCP Performance Enhancing Proxy to
Support Non-interactive Applications}

\begin{document}
\uiomasterfp[program={Informatics: Programming and System Architecture}, supervisor={Michael Welzl}]

\tableofcontents

\chapter{Intro}

\chapter{Background}

\section{Future of wireless communication.}
\subsection{5G}
The future of wireless communication has seen a lot of improvements such as...
... highly increased bandwidth ... using Millimeter frequency bands ... but at the cost of Highly fluctuating bandwidth with wireless networks, especially with mmWave.\\A big problem with mWave communication is Line of sight blocking. This is were ...

\subsection{Buffer bloat. (Buffering)}
Large buffers, create buffer bloat, which is detrimental for latency sensitive applications. Preferred to drop packets and keep buffers small to avoid buffering time sensitive packets such as SYN packets.

Most focus has been on (helping?) latency sensitive applications. This thesis will explore non interactive applications where latency is not that critical and more buffering is acceptable and desirable.

\subsection{Non Interactive Applications}
Non-Interactive applications such as Web traffic, File transfers and Videos? can benefit from larger buffering, especially with fluctuating bandwidths. Being able to have packets buffered for when the bandwidth is high will decrease delay times. (need citation or prove it myself?) 

\section{TCP/IP}

Interactive traffic uses TCP? {source}
End to end argument.
TCP handshake, reduce RTTs but using TCP Fast Open.
\subsection{TCP Fast Open}
Short flows terminating in a few round-trips. Meaning the "bottleneck" is the required initial TCP handshake.\\
TCP Fast Open allows data being exchanged during the handshake. 

\subsection{0 RTT}
Allow "syn fowarding" with TCP Fast Open creating a 0RTT increase when connecting through a proxy.
0RTT Transport Converter ~\cite{rfc8803}.

\subsection{Congestion control}
Congestion controller domains (different congestion controllers.)~\cite{rfc5783}.
Wireless versus Wired networks, different congestion controllers needs.

\section{PEPs}
More logic inside the networks. Domain splitting and 0RTT. {cite Kristjon Ciko?}

\section{Kernel Modules}
LKM (Loadable Kernel Modules), "program" running inside the Linux kernel.
Userspace vs Kernel.
\subsection{System Calls}
Reduce over head from userspace -> kernel system calls.

\chapter{Implementation | Design}
\chapter{Evaluation}
\chapter{Conclusion}

\bibliography{myBib.bib}{}
\bibliographystyle{plain}
\end{document}s